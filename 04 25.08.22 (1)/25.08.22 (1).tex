%!TEX TX-program = xelatex
%
%%%%%%%%%%%%%%%%%%%%
%
% Title: GCSE Mathematics & Additional Mathematics for 25.08.22 (1)
% Author: Eason Shao, Mr Finch-Noyes
% Date: 25.08.22 (1)
% Institude: Oxford International College
% Email: eason.syc@icloud.com; yicheng_shao@oxcoll.com
% GitHub: https://github.com/EasonSYC
% GitHub Repository: https://github.com/EasonSYC/GCSE-Maths-Notes
%
%%%%%%%%%%%%%%%%%%%%

\documentclass[8pt]{article}
\usepackage{../allan-eason}
\usepackage{siunitx}

\usetikzlibrary{positioning}
\usetikzlibrary{svg.path}

\graphicspath{ {./images/} }

\newcommand{\Date}{25.08.22 (1)}
\newcommand{\Name}{Mathematics}
\newcommand{\Title}{\textcolor{allandarkblue}{\Name}\ \textcolor{allancyan}{\Date}\ Notes}

\newcommand{\Author}{Eason Shao, Mr Finch-Noyes}

\author{\Author}
\title{\Title}
\date{\Date}

\geometry{a4paper, scale=0.8}

\lhead{\Title}

\begin{document}

	\maketitle

	\tableofcontents

	\section{Sets}

		\prob In a group of \(20\) students, \(12\) study History, \(10\) study Geography, \(3\) study neither. How many study both?

		\solution Let \(x\) be the number of students who study both History and Geography. According to the \defiword{Inclusion-Exclusion Principle}

		\[\mathrm{n} (A\cup B) = \mathrm{n} (A) + \mathrm{n}(B) - \mathrm{n} (A \cap B),\]

		where \(\mathrm{n} (A \cup B) = 20 - 3 = 17, \mathrm{n}(A) = 12, \mathrm{n}(B) = 10, \mathrm{n}(A \cap B) = x\).

		Simplify, we have

		\[17 = 12 + 10 - x,\]

		therefore we have \(x = 5\), which means \(5\) people study both.\newline

		\defi \defiword{(Set Builder Notation)} The notation \(A = \left\{x: P\right\}\) means that the set \(A\) includes and only includes all possible \(x\) which satisfies \(P\).

		\exmp \exmpword{(Set Builder Notation)} What does \(\left\{x: 1 \leq x \leq 6, x \in \NN\right\}\) mean? All values (a set). How is it different to \(x \in \{1, 2, 3, 4, 5, 6\}\)? One value of \(x\).\newline

		\prob What is the difference between \(\left\{x: 1 < x < 2\right\}\) and \(x \in (1, 2)\)?

		\solution The first one is all possible values, while the second one is a notation which means a single \(x\) is within the interval \((1, 2)\).\newline

		\exmp \exmpword{(Set of Points)} Find the equation of the following line. Describe the set of coordinates of points on this line in set build notation.

		\[
		\begin{tikzpicture}
			\draw[black, ->] (-1, 0)--(8, 0) node[below] {\(x\)};
			\draw[black, ->] (0, -1)--(0, 5) node[right] {\(y\)};
			\draw[black] (-2, 4)--(0, 3)--(6, 0)--(8, -1) node at (0, 3) [anchor = south west] {\((0, 3)\)} node at (6, 0) [anchor = north east] {\((6, 0)\)};
		\end{tikzpicture}
		\]

		The line is \(y = -\frac{1}{3}x + 3\); the set will be \(\left\{(x, y): y = -\frac{1}{3}x + 3\right\}\).

	\section{Unit Conversion}
		\subsection{General Unit Conversion}
			\exmp \exmpword{(Unit Conversion, Length)} \(\qty{1}{\metre} = \qty{1e2}{\cm} = \qty{1e3}{\mm}\).
			
			\exmp \exmpword{(Unit Conversion, Squared, Hectares)} \(\qty{1}{\metre\squared} = \qty{1e4}{\cm\squared} = \qty{1e6}{\mm\squared} = \qty{1e-4}{\hectare}\).
			
			\exmp \exmpword{(Unit Conversion, Cubed)} \(\qty{1}{\metre\cubed} = \qty{1e6}{\cm\cubed} = \qty{1e9}{\mm\cubed}\).

			\exmp \exmpword{(Unit Conversion, Km)} \(\qty{1}{\km} = \qty{1e3}{\metre}\).

			\exmp \exmpword{(Unit Conversion, Grams)} \(\qty{1}{\kg} = \qty{1e3}{\gram} = \qty{1e-3}{\tonne}\).

			\exmp \exmpword{(Unit Conversion, Litres)} \(\qty{1}{\litre} = \qty{1e-3}{\metre\cubed} = \qty{1e3}{\milli\litre} \qty{1e2}{\centi\litre}\).

		\subsection{Time Conversion}
			\prob A plane leaves London at \(21:15\). It lands in Singapore at \(17:45\). Given Singapore is \(7\) hours ahead, find the flight time.

			\solution \(21:15_{\text{London}} = 28:15_{\text{Singapore}} = 4:15_{\text{Singapore, +1}}\).
			
			\(17:45_{\text{Singapore}} - 4:15_{\text{Singapore} = 13:30}\). \(13\) hours and \(30\) minutes.

		\subsection{Bounds}
			\exmp \exmpword{(Significant Figures)} \(x = 47\) to \(2\) significant figures, so we have \(x \in [46.5, 47.5)\).

			\defi \defiword{(Lower Bound)} The lower bound for the previous example is \(46.5\).

			\defi \defiword{(Upper Bound)} The upper bound for the previous example is \(47.5\).\newline

			\prob \(a = 23.10\) to \(2\) decimal points. Bounds.

			\solution Lower bound is \(23.095\), upper bound is \(23.105\).\newline

			\prob \(b=700\) to \(2\) significant figures. Bounds.

			\solution Lower bound is \(695\), upper bound is \(705\).\newline

			\prob State the upper and lower bounds of

			\begin{enumerate}[label=\probword{(\arabic*)}]
				\item \(a+b\);
				
				\solution Lower: \(23.095 + 695\); Upper: \(23.105 + 705\).

				\item \(b-a\);
				
				\solution Lower: \(695 - 23.105\); Upper: \(705 - 23.095\).

				\item \(ab\);
				
				\solution Lower: \(695 \times 23.095\); Upper: \(705 \times 23.105\).

				\item \(\frac{b}{a}\).
				
				\solution Lower: \(695 / 23.105\); Upper: \(705 / 23.095\).
			\end{enumerate}

\end{document}