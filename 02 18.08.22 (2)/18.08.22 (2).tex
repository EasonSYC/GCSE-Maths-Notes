%!TEX TX-program = xelatex
%
%%%%%%%%%%%%%%%%%%%%
%
% Title: GCSE Mathematics & Additional Mathematics for 18.08.22 (2)
% Author: Eason Shao, Mr Finch-Noyes
% Date: 18.08.22 (2)
% Institude: Oxford International College
% Email: eason.syc@icloud.com; yicheng_shao@oxcoll.com
% GitHub: https://github.com/EasonSYC
% GitHub Repository: N/A
%
%%%%%%%%%%%%%%%%%%%%

\documentclass[8pt]{article}
\usepackage{../allan-eason}

\usetikzlibrary{positioning}
\usetikzlibrary{svg.path}

\graphicspath{ {./images/} }

\newcommand{\Date}{18.08.22 (2)}
\newcommand{\Name}{Mathematics}
\newcommand{\Title}{\textcolor{allandarkblue}{\Name}\ \textcolor{allancyan}{\Date}\ Notes}

\newcommand{\Author}{Eason Shao, Mr Finch-Noyes}

\author{\Author}
\title{\Title}
\date{\Date}

\geometry{a4paper, scale=0.8}

\lhead{\Title}

\begin{document}

	\maketitle

	\tableofcontents

	\section{Sequence}

		\exmp \exmpword{(Sequence)} Find the next two items: \(2.1, 2.01, 2.001, 2.0001, \cdots\) \(2.00001, 2.000001\).

		\exmp \exmpword{(Sequence)} Find the next two items: \(1, 1/2, 1/3, 1/4, 1/5, \cdots\). \(1/6, 1/7\).\newline

		\prob Find the next two items: \(2/64, 5/32, 8/16, 11/8, \cdots\).
		
		\solution The numerator \(+3\), the denominator \(\div 2\). Therefore the answer is: \(14/4, 17/2\).\newline

		\prob Find the next two items: \(1, 1, 2, 3, 5, 8, 13, 21, \cdots\).
		
		\solution Every item is the sum of the previous two itmes. Therefore the answer is: \(13 + 21 = 34, 21 + 34 = 55\).

		\subsection{Mathematical Notation}

			\prob \(u_n = 4n^2 - 1\) defines a sequence.
			
			\begin{enumerate}[label=\probword{(\arabic*)}]
				\item Find \(u_1, u_2, u_3\).
				
				\solution \(u_1 = 4 \times 1^2 - 1 = 3, u_2 = 4 \times 2^2 - 1 = 15, u_3 = 4 \times 3^2 - 1 = 35\).

				\item What is \(n\) when \(u_n = 1295\)?
				
				\solution We have \(u_n = 4n^2 - 1 = 1295\), thus \(n^2 = 324\), and \(n = 18\).

				\item Show \(900\) is not in the sequence.
				
				\solution \(u_n = 900 = 4n^2 - 1 \Leftrightarrow 4n^2 = 901 \Leftrightarrow n^2 = 225.25 \notin \NN\), while \(n \in \NN\).
			\end{enumerate}

		\subsection{Other Way Around}

			\prob Find the \(n\)th term for \(2, 16, 54, 128, \cdots\).
			
			\solution 
			
			\begin{align*}
				2, 16, 54, 128, \cdots &\xrightarrow{\div 2} 1, 8, 27, 64, \cdots\\
				&= n^3.
			\end{align*}
			
			\(u_n = 2 \times n^3\).\newline

			\meth \methword{(Important Sequences)} Remember these sequences:

			\begin{center}
				\begin{tabular}{r||c|c|c|c|c}
					\(n\) & \(1\) & \(2\) & \(3\) & \(4\) & \(5\)\\
					\hline
					Square Numbers \(n^2\) & \(1\) & \(4\) & \(9\) & \(16\) & \(25\)\\
					\hline
					Cube Numbers \(n^3\) & \(1\) & \(8\) & \(27\) & \(64\) & \(125\)\\
					\hline
					Triangular Numbers \(n(n+1)/2\) & \(1\) & \(3\) & \(6\) & \(10\) & \(15\)\\
					\hline
					All Below: Exponential \(2^n\) & \(2\) & \(4\) & \(8\) & \(16\) & \(32\)\\
					\hline
					\(3^n\) & \(3\) & \(9\) & \(27\) & \(81\) & \(243\)\\
					\hline
					\(0.1^n\) & \(0.1\) & \(0.01\) & \(0.001\) & \(0.0001\) & \(0.00001\)\\
					\hline
					\((-1)^n\) & \(-1\) & \(1\) & \(-1\) & \(1\) & \(-1\)
				\end{tabular}
			\end{center}

			\prob Find the \(n\)th term for \(0, 2, 5, 9, 14, \cdots\).
			
			\solution
			
			\begin{align*}
				0, 2, 5, 9, 14, \cdots &\xrightarrow{+1} 1, 3, 6, 10, 15, \cdots\\
				&= \frac{n(n+1)}{2}.
			\end{align*}
			
			\(u_n = n(n+1)/2-1 = (n^2 + n - 2)/2\).\newline

			\prob Find the \(n\)th term for \(4, 9, 16, 25, 36, \cdots\).
			
			\solution \(u_n = (n-1)^2 = n^2 - 2n + 1\).\newline

			\prob Find the \(n\)th term for \(8, 16, 24, 32, 40, \cdots\).
			
			\solution
			
			\begin{align*}
				8, 16, 24, 32, 40, \cdots &\xrightarrow{\div 8} 1, 2, 3, 4, 5, \cdots\\
				&=n.
			\end{align*}
			
			\(u_n = 8n\).\newline

			\prob Find the \(n\)th term for \(3, 12, 27, 48, 75, \cdots\).
			
			\solution
			
			\begin{align*}
				3, 12, 27, 48, 75, \cdots &\xrightarrow{\div 3} 1, 4, 9, 16, 25, \cdots\\
				&= n^2.
			\end{align*}

			\(u_n = 3 n^2\).\newline

			\prob Find the \(n\)th term for \(1, 0.1, 0.01, 0.001, 0.0001, \cdots\).
			
			\solution \(u_n = 10 \times 0.1^{n} = 0.1^{n-1} = 10^{-n+1}\).\newline

			\prob Find the \(n\)th term for \(5, 0.5, 0.05, 0.005, \cdots\).
			
			\solution \(u_n = 50 \times 0.1^{n} = 5 \times 0.1^{n-1} = 5 \times 10^{-n+1}\).\newline

			\prob Find the \(n\)th term for \(3, 4, 11, 30, 67,\cdots\).
			
			\solution
			
			\begin{align*}
				3, 4, 11, 30, 67, \cdot &\xrightarrow{-3} 0, 1, 8, 27, 64\\
				&= (n-1)^3.
			\end{align*}
			
			\(u_n = (n-1)^3 + 3 = n^3 - 3n^2 + 3n + 2\).\newline

			\meth \methword{(Trying Out)} Try out after writing the formulae.\newline

			\prob Find the \(n\)th term for \(5, 20, 45, 80, 125, \cdots\).
			
			\solution
			
			\begin{align*}
				5, 20, 45, 80, 125, \cdots &\xrightarrow{\div 5} 1, 4, 9, 16, 25, \cdots\\
				&= n^2.
			\end{align*}
			
			\(u_n = 5 n^2\).\newline

			\prob Find the \(n\)th term for \(1, 15, 53, 127, 249, \cdots\).
			
			\solution
			
			\begin{align*}
				1, 15, 53, 127, 249, \cdots &\xrightarrow{+1} 2, 16, 54, 128, 250, \cdots\\
				&\xrightarrow{\div 2} 1, 8, 27, 64, 125, \cdots\\
				&= n^3.
			\end{align*}
			
			\(u_n = 2n^3 - 1\).\newline

			\prob Find the \(n\)th term for \(3, 5, 9, 17, 33, \cdots\).
			
			\solution
			
			\begin{align*}
				3, 5, 9, 17, 33, \cdots &\xrightarrow{-1} 2, 4, 8, 16, 32, \cdots\\
				&= 2^n. 
			\end{align*}
			
			\(u_n = 2^{n} + 1\).\newline

			\prob Find the \(n\)th term for \(2, 6, 12, 20, 30, \cdots\).
			
			\solution

			\begin{align*}
				2, 6, 12, 20, 30, \cdots &\xrightarrow{\div 2} 1, 3, 6, 10, 15, \cdots\\
				&= \frac{n(n+1)}{2}.
			\end{align*}
			
			\(u_n = n(n+1) = n^2 + n\).\newline

			\prob Find the \(n\)th term for \(1/3, 2/7, 4/11, 8/15, 16/19, \cdots\).
			
			\solution Numerator: \(2^n\); Denominator: \(4n-1\). \(u_n = 2^n / (4n-1)\).\newline

			\prob Find the \(n\)th term for \(0.1, -0.8, 2.7, -6.4, 12.5, \cdots\).
			
			\solution 
			
			\begin{align*}
				0.1, -0.9, 2.7, -6.4, 12.5, \cdots &\xrightarrow{\times 10} 1, -9, 27, -64, 125, \cdots\\
				&\xrightarrow{\div (-1)^{n+1}} 1, 9, 27, 64, 125, \cdots.
			\end{align*}
			
			 \(u_n = 0.1(-1)^{n+1}n^3\).\newline

			\prob Find the \(n\)th term for \(-2, 4, -8, 16, -32, \cdots\).
			
			\solution 
			
			\begin{align*}
				-2, 4, -8, 16, -32, \cdots &\xrightarrow{\div (-1)^n} 2, 4, 8, 16, 32, \cdots\\
				&= 2^n.
			\end{align*}

			\(u_n = (-2)^n\). \newline

			\prob Find the \(n\)th term for \(0.01, 0.04, 0.09, 0.16, 0.25, \cdots\).
			
			\solution
			
			\begin{align*}
				0.01, 0.04, 0.09, 0.16, 0.25, \cdots &\xrightarrow{\times 100} 1, 4, 9, 16, 25\\
				&= n^2.
			\end{align*}

			\(u_n = 0.01 n^2\).

		\subsection{Using Difference}

			\meth \methword{(Using Difference)}
			
			\begin{enumerate}[label=\methword{(\arabic*)}]
				\item Constant difference \(\Rightarrow\) Highest power is one;
				\item Difference is constant difference \(\Rightarrow\) Highest power is two;
				\item Third difference is same \(\Rightarrow\) Highest power is three;
				\item \(n\)th difference is same \(\Rightarrow\) Highest power is \(n\).
			\end{enumerate}

			\prob Find the \(n\)th term for \(0, 6, 24, 60, 120, \cdots\).
			
			\solution First Difference: \(6, 18, 36, 60\). Second Difference: \(12, 18, 24\). Third Difference: \(6, 6\). Therefore it is cubic.

			Study sequence \(n^3\). \(1, 8, 27, 64, 125\). First Difference: \(7, 19, 37, 61\). Second Difference: \(12, 18, 24\). Same as the sequence. Therefore the second highest is linear.

			\begin{align*}
				0, 6, 24, 60, 120, \cdots &\xrightarrow{-n^3} -1, -2, -3, -4, -5, \cdots\\
				&\xrightarrow{\times -1} 1, 2, 3, 4, 5, \cdots\\
				&= n.
			\end{align*}

			\(u_n = n^3 - n\).

\end{document}