%!TEX TX-program = xelatex
%
%%%%%%%%%%%%%%%%%%%%
%
% Title: GCSE Mathematics & Additional Mathematics for 06.09.22
% Author: Eason Shao, Mr Finch-Noyes
% Date: 06.09.22
% Institude: Oxford International College
% Email: eason.syc@icloud.com; yicheng_shao@oxcoll.com
% GitHub: https://github.com/EasonSYC
% GitHub Repository: https://github.com/EasonSYC/GCSE-Maths-Notes
%
%%%%%%%%%%%%%%%%%%%%

\documentclass[8pt]{article}
\usepackage{../allan-eason}

\usetikzlibrary{positioning}
\usetikzlibrary{svg.path}

\graphicspath{ {./images/} }

\newcommand{\Date}{06.09.22}
\newcommand{\Name}{Mathematics}
\newcommand{\Title}{\textcolor{allandarkblue}{\Name}\ \textcolor{allancyan}{\Date}\ Notes}

\newcommand{\Author}{Eason Shao, Mr Finch-Noyes}

\author{\Author}
\title{\Title}
\date{\Date}

\geometry{a4paper, scale=0.8}

\lhead{\Title}

\begin{document}

	\maketitle

	\tableofcontents

	\section{Proportion}

		\defi \defiword{(Direct Proportion)} \(x\) and \(y\)are directly proportional states that \(y = kx\), using the symbol \(x \propto y\). Another term used is \(x\) and \(y\) \defiword{vary directly}.

		\[
			\begin{tikzpicture}
				\draw[black, ->] (-1, 0)--(5, 0);
				\draw[black, ->] (0, -1)--(0, 5);
				\draw[black] (0, 0)--(4, 4);
			\end{tikzpicture}
		\]

		Whatever we \(\times, \div\) \(y\) by, we do the same to \(x\).

		\defi \defiword{(Inversely Proportion)} \(x\) and \(y\) are inversely proportional to \(y\) states that \(x \propto 1/y\), same as \(y = k/x\). Another term used is \(x\) and \(y\) \defiword{vary indirectly}.

		\[
			\begin{tikzpicture}
				\draw[black, ->] (-1, 0)--(5, 0);
				\draw[black, ->] (0, -1)--(0, 5);
				\draw[black, domain = 0.25:4, samples = 1000] plot (\x, {1 / \x});
			\end{tikzpicture}
		\]

		Whatever we \(\times y\) by, we do \(\div\) to \(x\).

		Whatever we \(\div y\) by, we do \(\times\) to \(x\).

		\exmp \exmpword{(Proportion)} \(6\) workers take \(20\) days to build \(15\) products. Fill in the table below: (Do assume that products is directly proportional to workers and days)

		\begin{center}
			\begin{tabular}{c|c|c}
				Workers & Days & Products\\
				\hline\hline
				\(6\) & \(20\) & \(15\)\\
				\hline
				\(6\) & \(4\) & \exmpword{\(3\)}\\
				\hline
				\(16\) & \(20\) & \exmpword{\(40\)}\\
				\hline
				\(18\) & \(100\) & \exmpword{\(225\)}\\
				\hline
				\(2\) & \exmpword{\(60\)} & \(15\)\\
				\hline
				\exmpword{\(30\)} & \(4\) & \(15\)\\
				\hline
				\(30\) & \exmpword{\(16\)} & \(60\)\\
				\hline
				\exmpword{\(0.6\)} & \(40\) & \(3\)\\
				\hline
				\exmpword{\(4\)} & \(20\) & \(10\)\\
				\hline
				\exmpword{\(1.2\)} & \(20\) & \(3\)\\
				\hline
				\(6\) & \(10\) & \exmpword{\(7.5\)}\\
			\end{tabular}
		\end{center}

	\prob The square root of \(a\) or \(\sqrt{a}\) is inversely proportional to \(b^3\). When \(a=16\), \(b=10\).

	\begin{enumerate}[label=\probword{(\arabic*)}]
		\item Find an equation linking \(a\) and \(b\).
		
		\solution \(\sqrt{a} \propto b^{-3}\), therefore \(a \propto b^{-6}\). \(a = k \times b^{-6},\) and \(16 = k \times 10^{-6}\). Therefore \(k = 1.6 \times 10^7\) and \(a = 1.6 \times 10^7 b^{-6}\).

		\item Find \(b\) when \(a=36\).
		
		\solution Too difficult to calculate, skipped.
	\end{enumerate}

\end{document}